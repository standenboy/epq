\documentclass[a4paper,12pt]{article}

\usepackage{biblatex}
\usepackage{geometry}
\usepackage{titling}
\usepackage{titlesec}
\usepackage[english]{babel}
\usepackage[hidelinks]{hyperref}
\usepackage{listings}
\usepackage{xcolor}
\usepackage{graphicx}
\usepackage{forest}
\usepackage{tikz-qtree}

\addbibresource{ref.bib}

\titleformat{\section}
{\Huge}
{}
{0em}
{}[\titlerule]
\geometry{
        a4paper,
        total={170mm,257mm},
        left=25mm,
        right=25mm,
}

\author{Lucas Standen}
\title{WORKING TITLE: Why FOSS software is preferred in the 
development and privacy space?}


\begin{document}
\maketitle

\newpage
\tableofcontents
\newpage

{\setlength{\parindent}{0cm}

\section{Used language in this paper}
Throughout this paper I will use language specific to the field of computer science, and as such
it makes sense to provide a brief overview for those who don't know what specific terms mean.
\begin{description}
	\item[Licenses] In this setting a license is a legal document that is distributed with
		almost all modern software, which describes how someone can use a piece of software
	\item[Free Software] This term refers to software under specific licenses, making them 
		free for the user to use (free as in freedom, not the monetary cost). This will
		be covered further in the next section.
	\item[Open Source] This term refers to a piece of software, where the original code for it
		is publicly available. This too will be covered further in the next section.
	\item[FOSS] An acronym for "Free and Open Source Software".
\end{description}

\section{What is Free Software?}
The free software movement is one that has been active for over 40 years \cite{GNUmaifesto}, it has
created some of the most important tools in computing that are used by billions on a daily basis. 
It is so engraved in our lives, yet so few even know what the term means; In a simple note, it is
software for a computer, phone or other device that can be used without violating the users 
freedom.

The definition of what violates a users freedom can vary depending on who you ask, but it was
originally said as follows:
\begin{description}
	\item[0] The freedom to run a program for any purpose
	\item[1] The freedom to study how a program works, and modify it to your needs
	\item[2] The freedom to redistribute a piece of software
	\item[3] The freedom to redistribute a edited version of software publicly
\end{description}
\cite{FREESOFTWAREDEFINTION}
\section{A brief history of FOSS software}
\section{Examples of free software}
\section{Comparing free software to its proprietary counterparts}
\section{What makes free software so appealing to developers?}
\section{What makes free software so appealing to privacy experts?}
\section{Where else is free software used and why?}
\section{What's next for the free software space?}
\section{Final thoughts}

\newpage
\printbibliography
}
\end{document}
