\documentclass[a4paper,12pt]{article}

\usepackage[backend=bibtex]{biblatex}
\usepackage{geometry}
\usepackage{titling}
\usepackage{titlesec}
\usepackage[english]{babel}
\usepackage[hidelinks]{hyperref}
\usepackage{listings}
\usepackage{xcolor}
\usepackage{graphicx}
\usepackage{forest}
\usepackage{tikz-qtree}
\usepackage{setspace}

\addbibresource{ref.bib}

\titleformat{\section}
{\Huge}
{}
{0em}
{}[\titlerule]
\geometry{a4paper,total={170mm,257mm},left=25mm,right=25mm,}

\author{Lucas Standen}
\title{WORKING TITLE: Why FOSS software is preferred in the 
development and privacy space?}


\begin{document}
\maketitle

\newpage

\section{Using this document}
This document is written using the {\LaTeX} text compiler. The compiler has set up clickable links,
clickable references and a clickable table of contents, so please use these to your advantage. 
The Tex source and Bib Tex bibliography is available for all at 
\url{https://github.com/standenboy/epq/}.

\tableofcontents
\newpage

\setlength{\parskip}{1em}

{\setlength{\parindent}{0cm}

\section{A brief introduction}

\section{Used language in this paper}
Throughout this paper I will use language specific to the field of computer science, and as such
it makes sense to provide a brief overview for those who don't know what specific terms mean.

\begin{description}
	\item[Licenses] In this setting a license is a legal document that is distributed with
		almost all modern software, which describes how someone can use a piece of software
	\item[Free Software] This term refers to software under specific licenses, making them 
		free for the user to use (free as in freedom, not the monetary cost). This will
		be covered further in the next section.
	\item[Open Source] This term refers to a piece of software, where the original code for it
		is publicly available. This too will be covered further in the next section.
	\item[FOSS] An acronym for "Free and Open Source Software".
\end{description}

\section{What is Free Software?}
The Free Software movement is one that has been active for over 40 years \cite{GNUmaifesto}, it has
created some of the most important tools in computing that are used by billions on a daily basis. 
It is so engraved in our lives, yet so few even know what the term means; In a simple note, it is
software for a computer, phone or other device that can be used without violating the users 
freedom.

The definition of what counts Free Software and what is software freedom can vary depending on who 
you ask, but it was originally written that software that allows the following freedoms is 
Free Software:

\begin{description}
	\item[0] The freedom to run a program for any purpose
	\item[1] The freedom to study how a program works, and modify it to your needs
	\item[2] The freedom to redistribute a piece of software
	\item[3] The freedom to redistribute a edited version of software publicly
\end{description}
\textit{These freedoms were written by Richard Stallman\cite{FOSSdef} who is ever 
	important in this space.}

It is important that one does not confuse Free Software with software that is monetarily free, 
this is known as Freeware. Free Software defends the users rights to use and modify software and
is not focused on its cost.

One should also note the differences between Free Software and Open Source software. In Open Source
software, like Free Software, the original code for a program is available to anyone, however
in Open Source, this is to better the projects development and usability, whereas in Free Software
it is to better the users freedom. They both use the same methods to achieve differing goals; this
often leads them to be commonly used together, as the benefits a user gets from Free Software is 
much the same in Open Source software, and vice versa.

The main goal of Free Software is to allow the user to have as much freedom as possible when using 
a piece of software for any purpose. This is in contrast to the traditional alternative, called
Proprietary Software, which can be defined as software that the user can not edit, modify or 
redistribute without the original publishers permission. This kind of software intentionally 
restricts the users freedom, usually for the purpose of profit or control of the software. Some 
common examples of Proprietary Software, are Microsoft's \textit{Windows}, Apple's \textit{iOS}, 
and Google's \textit{Chrome} web browser.

Many people don't know that they already use Free Software\cite{COMMONfoss}, but often the tools
they use most often are Free Software. A few examples of this are, Krita\cite{KRITA}; a graphics 
design and art tool that is used frequently in animation, and other digital art, is made and 
managed by the KDE foundation\cite{KDE}, who make exclusively Free Software. Dovecot\cite{DOVECOT}; 
an email server which some major email providers use and is Free Software and commonly used, 
 A final example is Firefox\cite{FIREFOX} a Free Software web browser made by Mozilla that 
makes up 2.71\% of the browser market share as of 2024, however in the past has had up to 
30\%\cite{BROWSERmarketshare}.These are all more modern examples of Free Software, however over 
the past 40 years, there have been countless others. 
\section{A brief history of FOSS}
The term Free Software was first coined by Richard Stallman in 1983\cite{GNUproject}, however even
before this, examples of Free Software (and the disapproval of Proprietary Software), were already
starting to show. 

One of the earliest examples of this, was Microsoft's \textit{An open letter to hobbyists}, which
was written by Bill Gates in 1976. This letter detailed that people had been stealing from 
Microsoft, as many people had brought hardware through them, but far fewer people had brought the
software. The fact this was happening at a scale large enough to cause this showed how many 
computing groups, also known as hacker groups/spaces, weren't willing to pay for the software they
used, believing that if they brought the hardware they had done all that was needed\cite{OPENletter}. 
It is often believed that this is one of the first examples of \textit{hacker culture}, which 
would become more common into the 80's and 90's.

A key figure in \textit{hacker culture}, as previously mentioned, was Richard Stallman. In the 
early 1980's he left his job at MIT to work full time on the GNU project, which was designed
to be a full recreation of AT\&T's Unix operating system from the ground up. The idea was to allow
anyone access to a Unix like machine without paying AT\&T's expensive license fees, and allow any
user to view it, redistribute or edit; it was to be the first fully free operating system. The
early development of GNU was relatively slow, and it was not a completely free system for many
years, as some core parts of the operating system were missing, meaning non-free alternatives had 
to be used. However this would later change in 1991.

In 1988 BSD Net1 would release\cite{BSDnet1}, this was the first fully open version of the Berkeley 
Software Distribution version of Unix. It had completely rewritten all the code from the original 
Unix that previous versions contained, meaning it was now completely free from AT\&T's licenses.
It would be the start of a long linage of open source operating systems which are now the base
of MacOS, FreeBSD and OpenBSD.

The GNU project, while still not fully finished, saw the final piece of the puzzle when 
Linux\cite{LINUX} released in 1991, it was a fully open kernel which GNU was still lacking (however 
it did get its own kernel called GNU hurd but Linux is far more commonly used). With GNU and Linux 
paired together a user could finally get a fully free operating system for general use, this 
combination of software is still in use today, having a 4.7\% market share globally on desktop
computers\cite{LINUXmarket}, And on web servers it is dominant. In recent years it has also shown
some use in gaming, with it being the operating system used by Valves \textit{steam deck} gaming 
handheld\cite{STEAMdeck}.

Since Linux's release there haven't been as many major events, however there has been a slow tick
in development, with a large jump over Covid, with Free Software now being completely usable against
its Proprietary counterpart.

\section{Examples of Free Software}
\section{Comparing Free Software to its proprietary counterparts}
\section{What makes Free Software so appealing to developers?}
\section{What makes Free Software so appealing to privacy experts?}
\section{Where else is Free Software used and why?}
\section{What's next for the Free Software space?}
\section{Final thoughts}

\newpage
\printbibliography
}
\end{document}
